\documentclass{article}
\usepackage[utf8]{inputenc}
\usepackage[margin=1in]{geometry}

\title{CSoI Mid-Year Progress Report\\
The Polarization of Information}
\author{Charles Dickens, Pamela Bilo Thomas, Ram Hari Dahal, Prajjwal Dangal}
\date{February 2019}

\begin{document}

\maketitle

\par \noindent \textbf{Team Communications}

\par For daily communications, the team has created a private work-space on the team collaboration tool, \textit{Slack}. This has been a efficient tool to post brief updates and stay in touch. 
\par Furthermore, the team has been meeting regularly via \textit{Zoom} chat. \textit{Zoom} is a video communications platform that is being used by the team for weekly face to face meetings. During these meetings the team discusses the progress made and plan the next steps.
\par The resources (including posters, papers, and presentation materials) and source code for the project are hosted on GitHub repositories under the Polarization of Information organization. GitHub also supports project boards, which our teams uses to plan and assigns tasks. This feature has proven to be very useful for keeping track of progress and organizing the team's work schedule. 

\vspace{0.5cm}

\par \noindent \textbf{Progress Towards Results}

\par Thus far, the team has implemented a semi-supervised approach to modeling a network of information sources and simultaneously identifying a hierarchy of potential communities. Then the polarization of the network is quantified using graph metrics.

\par Twitter granted the team access the premium search API sandbox for this project. Twitter Premium API allows for 25,000 tweets to be collected each month. The system has analyzed multiple topics of discussion on Twitter and has identified groups that align with intuition. The network structure including polarity calculations can be successfully measured and compared across topics. Results are showing that not only are tweets with similar sentiment being clustered by the model, but groups of tweets that are discussing and making similar points about a topic are identifiable.  

\par Currently the team is moving in the direction of analyzing information outside of Twitter. The team has developed a web scraper to collect news articles found on the web and is currently implementing a workflow for constructing a network model of news articles. Additionally, the team is making consistent efforts to make the methodology for modeling and community detection more robust and efficient by building on ideas from similar research. 

\vspace{0.5cm}

\par \noindent \textbf{Presentations and Posters}

\par A paper and poster for the The Polarization of Information project has been presented at the UH Manoa College of Engineering 496 Poster Session. Along with the poster, a brief slide show presentation of the results of the project as of December 2018 was made. These resources are available on The Polarization of Information GitHub organization's resource repository. 

\vspace{0.5cm}

\par \noindent \textbf{Goals}

\par The team intends to submit a paper to the The 2019 IEEE/ACM International Conference on Advances in Social Networks Analysis and Mining, ASONAM 2019. The deadline to submit is May 15th. 


\end{document}